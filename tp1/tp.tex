\documentclass[10pt, a4paper]{article}
\usepackage[paper=a4paper, left=1.5cm, right=1.5cm, bottom=1.5cm, top=1.5cm]{geometry}
\usepackage[latin1]{inputenc}
\usepackage[T1]{fontenc}
\usepackage[spanish]{babel}
\usepackage{indentfirst}
\usepackage{fancyhdr}
\usepackage{latexsym}
\usepackage{lastpage}
\usepackage{aed2-symb,aed2-itef,aed2-tad,caratula}
\usepackage[colorlinks=true, linkcolor=blue]{hyperref}
\usepackage{calc}

\titulo{Trabajo Pr\'actico de Especificaci\'on}

\materia{Algoritmos y Estructuras de Datos II}
\grupo{Grupo 1}
\integrante{B\'alsamo, Facundo}{874/10}{facundobalsamo@gmail.com}
\integrante{Lasso, Nicol\'as}{892/10}{lasso.nico@gmail.com}	
\integrante{Rodr\'iguez, Agust\'in}{120/10}{agustinrodriguez90@hotmail.com}
\integrante{Tripodi, Guido}{843/10}{guido.tripodi@hotmail.com}
\parskip=5pt % 10pt es el tama�o de fuente

\begin{document}
\maketitle


\section{TAD \tadNombre{LinkLinkIT}}

\begin{tad}{\tadNombre{LinkLinkIT}}
\tadGeneros{LinkLinkIT}
\tadExporta{LinkLinkIT, generadores, observadores}
\tadUsa{\tadNombre{ArbolCategorias}}
\tadIgualdadObservacional{l}{l'}{LinklinkIT}{}
\tadAlinearFunciones{categorias}{LinkLinkIT}
\tadAlinearFunciones{accesos}{LinkLinkIT/it,Link/l}
\tadAlinearFunciones{linksXCategoria}{LinkLinkIT/it,categoria/c}{}{}
\tadAlinearFunciones{agregarLink}{LinkLinkIT/it,Link/l,categoria/c}{}{}
\tadAlinearFunciones{$\#$LinksXCategoria}{LinkLinkIT/it,categoria/c}{}{}

\tadObservadores

\tadOperacion{categorias}{LinkLinkIT}{ArbolDeCategorias}{}
\tadOperacion{accesos}{LinkLinkIT/it,Link/l}{tupla($Fecha,Nat$)}{$l \in LinkLinkIT(it)$}
\tadOperacion{linksXCategoria}{LinkLinkIT/it,categoria/c}{conj(Link)}{$c \in dameTodos(categorias(it)$}

\tadGeneradores

\tadOperacion{iniciar}{ArbolCategorias}{LinkLinkIT}{}
\tadOperacion{agregarLink}{LinkLinkIT/it,Link/l,categoria/c}{LinkLinkIT}{$\neg( l \in links(it))$ $ \&\&    $ $ c \in dameTodos(categorias(it))$}
\tadOperacion{accederLink}{LinkLinkIT/it,Link/l,fecha/f}{LinkLinkIT}{$l \in links(it)$ $ \&\&    $ $ f >= ultimoAcceso(it)$}

\tadOtrasOperaciones

\tadOperacion{links}{LinkLinkIT}{conj(Link)}{} %//Devuelve todos los links agregados a un LinkLinkIT
\tadOperacion{accesos3Dias}{LinkLinkIT/it,Link/l}{nat}{$l \in links(it)$} %//Dado un link existente devuelve la cantidad de accesos en los ultimos 3 dias
\tadOperacion{$\#$LinksXCategoria}{LinkLinkIT/it,categoria/c}{nat}{$c \in dameTodos(categorias(it))$} %Devuelve la cantidad de links que hay en la familia de la categoria dada
\tadOperacion{ultimoAcceso}{LinkLinkIT}{Fecha}{} %Devuelve la ultima fecha en la que se realizo un acceso
	


\tadAxiomas[\paratodo{nat}{n, m}]
\tadAlinearAxiomas{pred(suc($n$))}
\tadAxioma{$0=0?$}{true}

\end{tad}

\section{TAD \tadNombre{ArbolDeCategorias}}

\begin{tad}{\tadNombre{ArbolDeCategorias}}
\tadGeneros{ArbolDeCategorias}
\tadExporta{ArbolDeCategorias, generadores, observadores}
\tadUsa{\tadNombre{ArbolCategorias}}
\tadIgualdadObservacional{ac}{ac'}{ArbolDeCategorias}{}
\tadAlinearFunciones{raiz}{ArbolDeCategorias}{}{}
\tadAlinearFunciones{hijos}{ArbolDeCategorias}{}{}
\tadAlinearFunciones{AC}{conj(ArbolDeCategorias)/a,categoria/c}{}{}
\tadAlinearFunciones{dameTodos}{ArbolDeCategorias}{}{}
\tadAlinearFunciones{losFamiliares}{categoria/c,ArbolDeCategorias/ac}{}{}
\tadAlinearFunciones{estaEnAlguno}{categoria,conj(ArbolDeCategorias)}{}{}

\tadObservadores

\tadOperacion{raiz}{ArbolDeCategorias}{categoria}{}
\tadOperacion{hijos}{ArbolDeCategorias}{conj(ArbolDeCategorias)}{}


\tadGeneradores

\tadOperacion{AC}{conj(ArbolDeCategorias)/a,categoria/c}{ArbolDeCategorias}{$\neg estaEnAlguno(c,a)$}

\tadOtrasOperaciones


\tadOperacion{dameTodos}{ArbolDeCategorias}{conj(categoria)}{} %Conjunto de todas las categorías del arbol
\tadOperacion{losFamiliares}{categoria/c,ArbolDeCategorias/ac}{conj(categoria)}{$c \in dameTodos(ac)$} %Dada una categoria en particular de un arbol, devuelve un conjunto con ella y todas las categorias que son familia
\tadOperacion{estaEnAlguno}{categoria,conj(ArbolDeCategoria)}{bool}{} %Dado un conjunto de arboles, se fija si la categoria dada esta en alguno de esos


\tadAxiomas[\paratodo{nat}{n, m}]
\tadAlinearAxiomas{pred(suc($n$))}
\tadAxioma{$0=0?$}{true}

\end{tad}

\end{document}