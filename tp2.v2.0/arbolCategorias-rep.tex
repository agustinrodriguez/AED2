\begin{center}
\begin{tabular}{|l|} 
\hline
\\
ArbolCategorias \textbf{se representa con} estrAC, donde estrAC es \tupla{\\
\hspace*{6em}\param{}{raiz}{puntero(datosCat)},\\\hspace*{6em}\param{}{cantidad}{nat},\\
\hspace*{6em}\param{}{alturaMax}{nat},\\
\hspace*{6em}\param{}{familia}{diccTrie(Categoria, puntero(datosCat))},\\
\hspace*{6em}\param{}{categorias}{Lista(datosCat)}}\\ \\

\hspace*{6em}datosCat es \tupla{\\
\hspace*{6em}\param{}{categoria}{Categoria},\\ 
\hspace*{6em}\param{}{id}{nat},\\ 
\hspace*{6em}\param{}{altura}{nat},\\ 
\hspace*{6em}\param{}{hijos}{Conj(puntero(datosCat))},\\
\hspace*{6em}\param{}{padre}{puntero(datosCat)}}\\ \\ 
\hline
\end{tabular}
\end{center}

\subsubsection{Invariante de Representaci\'on}
\paragraph{El Invariante Informalmente}
\begin{enumerate}
\item Para cada '\textit{padre} obtener el significado devolvera un puntero(datosCat) donde '\textit{categoria}' es igual a la clave
\item Para toda clave '\textit{padre}' que exista en '\textit{familia}' debera ser o raiz o pertenecer a algun conjunto de punteros de '\textit{hijos}' de alguna clave '\textit{padre}'
\item Todos los elementos de '\textit{hijos} de una clave '\textit{padre}', cada uno de estos hijos tendran como '\textit{abuelo}' a ese '\textit{padre}' cuando sean clave.
\item '\textit{cantidad}' sera igual a la longitud de la lista '\textit{categorias}'.
\item Cuando la clave es igual a '\textit{raiz}' la '\textit{altura} es 1.
\item La '\textit{altura}' del puntero a datosCat de cada clave es menor o igual a '\textit{alturaMax}'.
\item Existe una clave en la cual, la '\textit{altura}' del significado de esta es igual a '\textit{alturaMax}'.
\item Los '\textit{hijos}' de una clave tienen '\textit{altura}' igual a 1 + '\textit{altura} de la clave.
\item Todos los '\textit{id}' de significado de cada clave deberan ser menor o igual a '\textit{cant}'.
\item No hay '\textit{id}' repetidos en el '\textit{familia}.
\item Todos los '\textit{id}' son consecutivos.

\end{enumerate}

\paragraph{El Invariante Formalmente}
\begin{Rep}{estrAC}{ac}
\repfunc{(\forall\param{}{x}{string}) (def?(x,e.familia)) \Leftrightarrow (*obtener(x,e.familia)).categoria = x}{\yluego}
\repfunc{(\forall\param{}{x,y}{string}) (def?(x,e.familia)) \Leftrightarrow (x == e.raiz) \vee (def?(y,e.familia)) \yluego \\ x \in hijosDe(*((obtener(y,e.familia))).hijos)}{\yluego}
\repfunc{(\forall\param{}{x,y}{string}) (def?(x,e.familia)) \land (def?(y,e.familia)) \Rightarrow_{L} \\ y \in *((obtener(x,e.familia)) ).hijos \Leftrightarrow \\ (*(*(obtener(y,e.familia))).padre).categoria = x}{\yluego}
\repfunc{e.cantidad  = longitud(e.categorias)}{\yluego}
\repfunc{(\forall\param{}{x}{string}) (def?(x,e.familia)) \land x = e.raiz \Rightarrow_{L} \\ (*(obtener(x,e.familia))).altura = 1}{\yluego}
\repfunc{(\forall\param{}{x}{string}) (def?(x,e.familia)) \Rightarrow_{L} (*obtener(x,e.familia)).altura \leq e.alturaMax}{\yluego}
\repfunc{(\exists\param{}{x}{string}) (def?(x,e.familia)) \yluego *((obtener(x,e.familia)) ).altura = e.alturaMax}{\yluego}
\repfunc{(\forall\param{}{x,y}{string}) (def?(x,e.familia)) \wedge (def?(y,e.familia)) \yluego \\ y \in hijosDe((*(obtener(x,e.familia))).hijos) \Rightarrow \\ (*(obtener(y,e.familia))).altura = 1 + (*(obtener(x,e.familia))).altura}{\yluego}
\repfunc{(\forall\param{}{x}{string}) (def?(x,e.familia)) \Rightarrow_{L} (*(obtener(x,e.familia))).id \leq e.cant}{\yluego}
\repfunc{(\forall\param{x,y}{string}) (def?(x,e.familia)) \land (def?(y,e.familia)) \Rightarrow_{L} \\ (*(obtener(x,e.familia))).id \neq (*(obtener(y,e.familia))).id}{\yluego}
\repfunc{(\forall\param{x}{string}) (def?(x,e.familia)) (\exists\param{}{y}{string}) (def?(y,e.familia)) \Leftrightarrow \\ (*(obtener(y,e.familia))).id \leq e.cantidad \land (*(obtener(x,e.familia))).id < e.cantidad \yluego \\(*(obtener(y,e.familia))).id = 1 + (*(obtener(x,e.familia))).id}{}
\end{Rep}
\newpage

\subsubsection{Funci\'on de Abstracci\'on}
\begin{ABS}{e}{estrAC}{ac}{acat}
\absfunc{categorias(ac) \igobs todasLasCategorias(e.categorias)}{\yluego}
\absfunc{raiz(ac) \igobs (*e.raiz).categoria}{\yluego}
\absfunc{(\forall\param{}{c}{Categoria})esta?(c,ac) \land c\neq raiz(ac) \Rightarrow_{L} \\
padre(ac,c) = (*(*(obtener(c,e.familia))).padre).categoria}{\yluego}
\absfunc{(\forall\param{}{c}{Categoria})esta?(c,ac) \Rightarrow_{L} id(ac,c) = (*(obtener(c,e.familia))).id}{}
\end{ABS}

\paragraph{Funciones auxiliares}
\paragraph*{}

\tadOperacion{todasLasCategorias}{secu(datosCat)}{conj(categoria)}{}
\tadAxioma{todasLasCategorias}{Ag((prim(cs)).categoria,fin(cs))}