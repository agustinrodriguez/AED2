\vspace*{1em}
\begin{enumerate}

\item\textbf{iCrearItLinks}
\par Crea un itPuntLinks con la lista que se pasa como parametro y se la asigna a res, esto demora O(1).
\par \textbf{Orden Total:} \textbf{O(1)}

\item\textbf{iHaySiguiente?}
\par Se llama a HaySiguiente del Iterador de Lista en O(1).
\par \textbf{Orden Total:} \textbf{O(1)}

\item\textbf{iSiguiente}
\par Se llama a Siguiente del Iterador de Lista en O(1). Se devuelve una referencia del elemento link de la tupla DatosLink.
\par \textbf{Orden Total:} \textbf{O(1)}

\item\textbf{iSiguienteCat}
\par Se llama a Siguiente del Iterador de Lista en O(1). Se devuelve una referencia del elemento catDLink de la tupla DatosLink.
\par \textbf{Orden Total:} \textbf{O(1)}

\item\textbf{iSiguienteCantidadAccesosDelLink}
\par Se llama a cantAccesosDesde del Iterador en O(1). Se devuelve el valor entero de la cantidad de accesos del link en la posicion actual del iterador.
\par \textbf{Orden Total:} \textbf{O(1)}

\item\textbf{iAvanzar}
\par Se llama a Avanzar del Iterador de Lista en O(1).
\par \textbf{Orden Total:} \textbf{O(1)}

\item\textbf{iEliminarSiguiente}
\par Se llama a eliminarSiguiente del Iterador de Lista en O(1).
\par \textbf{Orden Total:} \textbf{O(1)}

\item\textbf{iBuscarMax}
\par Dado un itPuntLinks y una fecha, iteramos llamando cada vez a cantAccesosDesde para el iterador y la fecha. Cada vez nos cuesta O(1) y lo hacemos una vez para cada iteracion. En total nos cuesta O(longitud(siguientes(it))).
\par En O(1) copiamos el iterador a res sólo si la llamada a cantAccesosDesde nos dio mayor a la que resultaba del anterior valor de res.
\par finalmente avanzar el iterador nos cuesta O(1) tambien.
\par \textbf{Orden Total:} \textbf{O(longitud(siguientes(it)))}

\item\textbf{iUltFecha}
\par Dado un itPuntLinks, iteramos pidiendo el día al acceso mas nuevo, para eso generamos un itAccesos a la ultima posicion de la lista de accesos en O(1). Luego pedimos la fecha de ese acceso tambien en O(1).
\par Y evaluamos en O(1) si es mayor a la fecha que teniamos guardada en el resultado. Cambiandola en caso de ser necesario en O(1).
\par Como lo hacemos para cada link de la primera lista, nos cuesta O(longitud(siguientes(it)))
\par \textbf{Orden Total:} \textbf{O(longitud(siguientes(it)))}

\item\textbf{iCantAccesosDesde}
\par Dado un itPuntLinks y una fecha, obtengo en O(1) un itAccesos para la lista de accesos del siguiente del iterador.
\par Luego voy iterando itAccesos y si la fecha se encuentra dentro de la fecha pasada y la fecha pasada menos dos, sumo la cantidad de accesos para ese acceso, todo en O(1).
\par Como la lista de accesos iterada tiene a lo sumo 3 elementos, podemos considerar que iterarla nos lleva tiempo costante, o sea O(1).
\par \textbf{Orden Total:} \textbf{O(1)}

\item\textbf{iEstaOrdenada?}
\par Dado un itPuntLinks y una fecha, iteramos llamando cada vez a cantAccesosDesde para el iterador y la fecha. Cada vez nos cuesta O(1) y lo hacemos una vez para cada iteracion. En total nos cuesta O(longitud(siguientes(it))).
\par Comparamos cantAccesosDesde con la variable aux en O(1) y si aux es menor, cambiamos res a false en O(1)
\par Actualizaar el aux con cantAccesosDesde y avanzar el iterador nos cuesta O(1) en ambos casos.
\par \textbf{Orden Total:} \textbf{O(longitud(siguientes(it)))}

\end{enumerate}